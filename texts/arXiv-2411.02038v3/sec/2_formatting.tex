
\section{Related Work}
VQ-VAE \cite{NIPS2017_7a98af17} is the pioneering work to encode data into discrete representations. Building on these developments, VQGAN \cite{Esser_2021_CVPR} combines VQ-VAE with adversarial networks to improve the perceptual quality of generated samples and establish a fundamental quantization protocol in latent generative models \cite{Rombach_2022_CVPR,yu2022scaling,team2024chameleon}. 
Many traditional VQ works focus on training a better discrete representation rather than codebook utilization to improve reconstruction performance. For example, RVQ \cite{Lee2022AutoregressiveIG} and MoVQ \cite{Zheng2022HighQualityPI} enhance the reconstruction details with multichannel quantization. 
However, these methods suffer from a critical issue of representation collapse, as they struggle to scale the codebook size beyond 10k entries, limiting their scalability. In response to this challenge, several approaches have been proposed recently. 
DALLE \cite{pmlr-v139-ramesh21a} employs the gumbel-softmax trick \cite{DBLP:conf/iclr/JangGP17} and stochastic sampling strategies to activate most codes during training. However, during inference, only a small subset of codes is utilized for quantization \cite{Zhang_2023_CVPR}. 
Some methods design complex optimization strategies to improve codebook utilization, such as stochastic quantization \cite{pmlr-v162-takida22a}, distribution penalty \cite{VQWasserstein,Xiao2023SCVAESC} and codebook reset \cite{zheng2023online}.
\citet{pmlr-v202-huh23a} proposes rescaling the vectors in the codebook during training to match the distributions in the latent space. VQGAN-FC \cite{yu2022vectorquantized} introduces a method to map latent vectors into a lower-dimensional space followed by $l_2$ normalization to alleviate representation collapse. FSQ \cite{mentzer2024finite} extends this idea by projecting representations into a reduced-dimensional space, where they are quantized into a small set of fixed values. LFQ \cite{yu2024language}, a variant of FSQ, uses binary values for quantized representations, thereby simplifying the encoding process. While these methods improve the codebook utilization, they do so at the cost of model capacity by significantly reducing the dimensionality of latent space (often to as low as 8), leading to worse performance compared to vanilla VQ models when the codebook size is small and representation collapse is not severe. Additionally, VQGAN-LC \cite{Zhu2024ScalingTC} proposes to initialize the codebook using features extracted from the pre-trained CLIP model to avoid representation collapse. However, the reliance on the pre-trained model limits the VQ model's ability to generalize to diverse datasets and results in a performance plateau as the codebook size increases. In contrast, our method, SimVQ, effectively addresses the representation collapse problem with a simple linear layer, without sacrificing model capacity or relying on external pre-trained models.


\section{Representation Collapse in VQ Models}

\subsection{Preliminaries}

A vector quantized model is typically a reconstructive encoder-decoder architecture that includes a vector quantization layer to convert continuous representations into discrete codes. For simplicity, we represent an image with a single random variable $x$. Formally, the encoder $f_{\theta}$ maps the input image into a latent space, producing a continuous representation $z_{e}=f_{\theta}(x) \in \mathbb{R}^d$. This representation is then quantized using a learnable codebook $\bm{C}=[q_1,\ldots,q_K]\in \mathbb{R}^{K\times d}$, where $q_i$ is a codebook vector. We define $\delta_{k}\in \{0,1\}^{1\times K}$ as a characteristic (one-hot) vector where only the $k$-th element is $1$, such that $q_k = \delta_{k}\bm{C}\in \mathbb{R}^{1\times d}$. The quantization layer selects the nearest codebook vector $q_k$ by minimizing the Euclidean distance between $z_e$ and the codebook entries \cite{NIPS2017_7a98af17}:
\begin{align}
    k = \arg\min_{j} \|z_{e}-q_j\|^2_2 = \arg\min_{j} \|z_{e}-\delta_{j}\bm{C}\|^2_2.
\end{align}
The selected vector $q_k$ is then passed to the decoder $g_{\phi}$ to reconstruct the input image. 

To enable gradient propagation through the non-differentiable characteristic vector $\delta_{k}$, the straight-through estimator (STE) \cite{bengio2013estimating} is applied. During the backward pass, the gradient of $z_q=\delta_k \bm{C}$ is copied to $z_e$ as follows,
\begin{align}
    z_q = \text{sg}(\delta_{k}\bm{C} - z_e) + z_e,\quad \Rightarrow \frac{\partial z_q}{\partial z_e}=1
\end{align}
where \texttt{sg} is the stop gradient operator, ensuring the gradient for $\delta_k \bm{C}$ is discarded during the backward pass.

The learning objective is the combination of a reconstruction loss and commitment loss that ensures that the encoder commits to an embedding and the encoder's output does not drift:
\begin{equation}
    \mathcal{L} = \log p(x|z_q) + \|\text{sg}(\delta_{k}\bm{C})-z_e\|^2_2 + \beta \|\delta_{k}\bm{C} - \text{sg}(z_e)\|^2_2,
\end{equation}
where $\log p(x|z_q)$ is typically the mean squared error (MSE) loss $\|x-g_{\phi}(z_q)\|^2_2$ for image and audio data.



\subsection{Disjoint Optimization of Codebook}

In VQ models, only the nearest code is selected and updated via gradient descent. Ideally, all codebook entries should be updated and utilized for decoding. However, experimental evidence shows that only a small fraction of the codebook gets updated and utilized, leading to what is known as the representation collapse problem \cite{roy2018theory}. To investigate the root cause of this issue, we provide a theoretical analysis of the optimization dynamics in VQ models.

Due to the use of the straight-through estimator (STE) for gradient propagation, the codebook $\bm{C}$ can only be updated through the gradient of the commitment loss, which is defined as:
\begin{equation}
    \mathcal{L}_{commit}(\bm{C})=\|z_e - \delta_k\bm{C}\|_2^2.
\end{equation}
The codebook $\bm{C}$ is updated according to the following equation, where $\eta$ is the learning rate:
\begin{align}
    &\bm{C}^{(t+1)} = \bm{C}^{(t)} + \eta \mathbb{E}_{z_e} \left[\frac{\partial \mathcal{L}_{commit}(\bm{C}^{(t)})}{\partial \bm{C}^{(t)}}\right] \\
    &= \bm{C}^{(t)} - \eta \mathbb{E}_{z_e} \left[\delta_k^T\delta_k\bm{C}^{(t)}\right] + \eta \mathbb{E}_{z_e} \left[\delta_{k}^{T}z_e\right]
\end{align}
where $\delta_k^T\delta_k$ is the Kronecker delta matrix, defined as:
\begin{equation}
    (\delta_k^T \delta_k)_{ij} =
\begin{cases}
1 & \text{if } i = j = k, \\
0 & \text{otherwise}.
\end{cases}
\end{equation}
All vectors in $\bm{C}$ will be updated and utilized if and only if the expectation $\mathbb{E}_{z_e}\left[\delta_k^T\delta_k \right]$ converges to the identity matrix. Unlike variational autoencoders (VAEs) \cite{Kingma2013AutoEncodingVB}, which enforce a Gaussian distribution on the latent space via a KL-divergence penalty, VQ models optimize $z_e$ towards the selected codebook vectors $\mathbb{E}_{z_e} \left[\delta_k^T\delta_k\bm{C}\right]$. At the same time, the selected codebook vectors are optimized towards the distribution of $z_e$, resulting in the same selected subset of vectors moving closer to $z_e$, somewhat akin to a cocoon effect. However, this disjoint optimization of the codebook leads to part of the codebook, specifically $(\bm{I} - \mathbb{E}_{z_e}\left[\delta_k^T\delta_k \right])\bm{C}$, remaining un-updated and underutilized once the optimization process begins. This phenomenon occurs because the optimization focuses only on a subset of codebook vectors, leaving other vectors stagnant.

This analysis reveals the fundamental cause of representation collapse in VQ models: the disjoint optimization process that updates only a subset of codebook vectors. This insight forms the basis for our proposed solution, SimVQ, which aims to address this issue by optimizing the entire latent space spanned by the codebook, rather than individual code vectors.



\section{Addressing Collapse with Latent Linear Transformation}

\subsection{Reparameterize Codes with Latent Basis}

Let $\bm{W}=\{\bm{w}_1,\ldots ,\bm{w}_n\}$ be a basis of a linear space. Any vector $\bm{v}$ in the space can be uniquely expressed as a linear combination of the basis vectors with coefficients $c_1,\ldots,c_n \in \mathbb{R}$:
\begin{equation}
    \bm{v} = c_1\bm{w}_1+\cdots+c_n\bm{w}_n = \bm{c}\bm{W}.
\end{equation}
%Because
Given the equivalence between $\bm{v}$ and $\bm{c}\bm{W}$ in the linear space, we can reparameterize each vector in the codebook of VQ models with a new basis matrix $\bm{W}\in \mathbb{R}^{d\times d}$. Specifically, the codebook $\bm{C}=\{\bm{c}_1,\ldots,\bm{c}_K\}$ can be reparameterized as:
\begin{equation}
    \{\bm{\hat{c}}_1\bm{W},\ldots,\bm{\hat{c}}_N\bm{W}\}=\bm{\hat{C}}\bm{W}\in \mathbb{R}^{K\times d}.
\end{equation}
This reparameterization introduces two components: the basis matrix $\bm{W}$ and the coefficient matrix $\bm{\hat{C}}$. In the following, we will discuss the optimization of both the basis matrix $\bm{W}$ and the coefficient matrix $\bm{\hat{C}}$. For simplicity, we slightly abuse $\bm{C}$ and $\bm{\hat{C}}$ below.




\subsection{Asymmetric Optimization Dynamics}

While it is commonly accepted that multiplying two linear matrices is equivalent to a single linear layer, we argue that the disjoint optimization problem of the codebook in VQ models can be addressed by linear transformation. In vanilla VQ models, only the codebook $\bm{C}$ is responsible for minimizing commitment loss, leading to the disjoint optimization problem where only the selected codes will be updated. 

In contrast, when the codebook is reparameterized as $\bm{C}\bm{W}$, both the basis $\bm{W}$ and the coefficient matrix $\bm{C}$ contribute to minimizing the commitment loss. The gradients $\frac{\partial \mathcal{L}}{\partial \bm{W}}$ and $\frac{\partial \mathcal{L}}{\partial \bm{C}}$ can simultaneously reduce the loss. As a result, the optimization of the reparameterized codebook can be divided into three scenarios:
\begin{itemize}
\item Updating $\bm{C}$ with $\bm{W}$ frozen: Only the \textbf{selected} codes adapt to the latent distribution of $z_e$, as depicted on Fig. \ref{fig:intro}(a). The vanilla VQ is a special case with $\bm{W}=\bm{I}$.
\item Updating $\bm{W}$ with $\bm{C}$ frozen: The \textbf{entire} codebook $\bm{C}\bm{W}$ adjusts to the latent distribution of $z_e$. The basis matrix $\bm{W}$ rotates and stretches the space as shown in Fig. \ref{fig:intro}(b).
\item Updating both $\bm{C}$ and $\bm{W}$: The selected subset of codes moves towards $z_e$ while the space spanned by $\bm{W}$ undergoes simultaneous rotation and stretching.
\end{itemize}

To highlight the difference in optimization between $\bm{C}$ and $\bm{CW}$, we conduct a toy experiment in a two-dimensional setting and visualize the optimization process in Fig. \ref{fig:optim1} and Fig. \ref{fig:optim2}. 

\subsubsection{Toy Examples}


\begin{figure}[t]
    \centering
    \begin{minipage}[b]{0.49\columnwidth}
        \centering
        \includegraphics[width=\columnwidth]{material/dynamic3.pdf}
    \end{minipage}
    %\hspace{0.05\textwidth}
    \begin{minipage}[b]{0.49\columnwidth}
        \centering
        \includegraphics[width=\columnwidth]{material/dynamic2.pdf}
    \end{minipage}
    \caption{(a): (left) The optimization trajectory of the objective $\|\bm{x}-\bm{q}\|^2_2$, which is the same as vanilla VQ. Only a small fraction of points are updated while others remain inactive. (b): (right) The optimization trajectory of the objective $\|\bm{x}-\bm{q}\bm{w}\|^2_2$ with $\bm{q}$ frozen, which is the same as SimVQ. All the points are updated towards targets $x$.}
    \label{fig:optim1}
\end{figure}


\begin{figure}[t]
    \centering
    \begin{minipage}[b]{0.47\columnwidth}
        \centering
        \includegraphics[width=\columnwidth]{material/dynamic1.pdf}
    \end{minipage}
    % \hfill
    \begin{minipage}[b]{0.51\columnwidth}
        \centering
        \includegraphics[width=\columnwidth]{material/dynamic11.pdf}
    \end{minipage}
    \caption{(a): (left) The optimization trajectory of the optimization objective: $\|\bm{x}-\bm{q}\bm{w}\|^2_2$ with both $\bm{q}$ and $\bm{w}$ unfrozen. (b): (right) The Frobenius norm of the projection matrix $\bm{w}$ and loss curves. The loss quickly converges to 0 with $\bm{w}$ almost unchanged.}
    \label{fig:optim2}
\end{figure}


\begin{algorithm}[t]
   \caption{Training Procedure %Pseudo-code 
   for SimVQ}
   \label{alg:code}
\begin{algorithmic}
\STATE \textbf{Input:} Encoder $f_{\theta}$, Decoder $g_{\phi}$, Codebook $\bm{C}\in \mathbb{R}^{K\times d}$, Linear projector matrix $\bm{W}_{\psi}$, commitment weight $\beta$. 
\STATE \textbf{Output:} Model parameters $\theta, \phi, \psi$ and Codebook $\bm{C}$.
\STATE Initialize Codebook with Gaussian distribution and \textbf{freeze} the parameter of Codebook $\bm{C}$;
\REPEAT
    \STATE Draw $x\sim p_{data}(\bm{x})$;
    \STATE $z_e=f_{\theta}(x)$;
    \STATE \textcolor{blue}{/* Replace $q_j$ in vanilla VQ with proposed $q_j\bm{W}_{\psi}$.}
    \STATE Nearest code search: $k = {\arg\min}_{j} \|z_e - \textcolor{blue}{q_j\bm{W}_{\psi}}\|^2_2$, where $q_j\in \bm{C}$; % or $\bm{C}\bm{W}_{\psi}$
    \STATE Straight Through Estimation: $z_q=\text{sg}(\textcolor{blue}{q_k\bm{W}_{\psi}} - z_e)+z_e$;
    \STATE $\hat{x}=g_{\phi}(z_q)$;
    \STATE Minimize $\mathcal{L}(\theta,\phi,\psi)$=$\text{MSE}(x,\hat{x})+\beta \|z_e-\text{sg}(\textcolor{blue}{q_k\bm{W}_{\psi}})\|^2_2+\|\text{sg}(z_e)-\textcolor{blue}{q_k\bm{W}_{\psi}}\|^2_2$;
\UNTIL{converged}
\end{algorithmic}
\end{algorithm}


We randomly sample two target points $\bm{x}$ from Gaussian distribution as follows:
\begin{equation}
    \bm{x}_1 \sim \mathcal{N}(
    \left(
    \begin{matrix}
    2\\
    2
    \end{matrix}
    \right), 
    \left(
    \begin{matrix}
    1 & 0\\
    0 & 1
    \end{matrix}
    \right)
    ),\quad \bm{x}_2 \sim \mathcal{N}(
    \left(
    \begin{matrix}
    -2\\
    -2
    \end{matrix}
    \right), 
    \left(
    \begin{matrix}
    1 & 0\\
    0 & 1
    \end{matrix}
    \right)
    ).
\end{equation}
Then we initialize $10$ learnable vectors $\bm{q}$ from a Gaussian distribution:
\begin{equation}
    \{\bm{q}_i\}_{i=1}^{10} \sim \mathcal{N}(
    \left(
    \begin{matrix}
    0\\
    0
    \end{matrix}
    \right), 
    \left(
    \begin{matrix}
    1 & 0\\
    0 & 1
    \end{matrix}
    \right)
    ),
\end{equation}
During training with gradient descent, we introduce perturbation noise $\mathcal{N}(0,0.01)$ to the targets. In Fig. \ref{fig:optim1}(a), the optimization objective is similar to vanilla VQ: $\|\bm{x}-\bm{q}\|^2_2$. Only the nearest points $\bm{q}_4$ and $\bm{q}_{10}$ are updated. In contrast, in Fig. \ref{fig:optim1}(b), the optimization objective $\|\bm{x}-\bm{q}\bm{w}\|^2_2$ is similar to SimVQ with the points reparameterized by a learnable latent basis $\bm{w}$ and $\bm{q}$ frozen, resulting in the entire codebook $\{\bm{q}\}_{i=1}^{10}$ %is
being \textit{jointly} updated.

\begin{remark}
The simultaneous optimization of the latent basis $\bm{w}$ and the coefficient matrix $\bm{q}$ \textbf{may} lead to the collapse. 
\end{remark}

We provide an example in Fig. \ref{fig:optim2}(a) where the optimization objective is $\|\bm{x}-\bm{q}\bm{w}\|^2_2$ with $\bm{q}$ unfrozen this time. In the training process, only the nearest point $\bm{q}_1$ and point $\bm{q}_{10}$ move towards the target point, while other points remain almost unchanged. We also visualize the loss curve in Fig. \ref{fig:optim2}(b). The optimization objective with both $\bm{q}$ and $\bm{w}$ unfrozen converges quickly, where the norm of basis $\bm{w}$ is much smaller than the objective with $\bm{q}$ frozen. This indicates that the disjoint optimization of the codebook persists: $\bm{q}$ can directly commit to the loss and dominate the optimization process, with $\bm{w}$ being ignored, leading to the collapse quickly.



\subsection{Joint Optimization of the Codebook}



We propose SimVQ by simply using a learnable basis $\bm{W}\in \mathbb{R}^{d\times d}$ to reparameterize the codebook such that the codebook %becomes
is transformed into $\bm{C}\bm{W}$. The pseudo-code for this approach is provided in Algorithm \ref{alg:code}. During training, we optimize only the latent basis matrix $\bm{W}$, while keeping the coefficient matrix $\bm{C}$ frozen. The commitment loss for SimVQ is defined as:
\begin{equation}
    \mathcal{L}_{commit}(z_e,q_k)=\|z_e-\delta_k\bm{C}\bm{W}\|^2_2.
\end{equation}
The vanilla VQ model is a special case of SimVQ, where the linear basis matrix $\bm{W}$ is fixed as the identity matrix $\bm{I}$. The update for $\bm{W}$ with learning rate $\eta$ is:
\begin{align}
    &\bm{W}^{(t+1)} = \bm{W}^{(t)} - \eta \frac{\partial \mathcal{L}_{commit}(z_e,\bm{q}_k)}{\partial \bm{W}^{(t)}}\\ &= (\bm{I}-\eta \mathbb{E}_{z_e} \left[ \bm{C}^T\delta_k^T\delta_k\bm{C}\right])\bm{W}^{(t)} + \eta \mathbb{E}_{z_e} \left[ \bm{C}^T\delta_k^{T}z_e\right].
\end{align}
The term $\mathbb{E} \left[\bm{C}^T \delta_k^T \delta_k \bm{C}\right]$ represents the expectation of the quadratic form, and simplifies to $\mathbb{E}[\bm{q}_k^T \bm{q}_k]$. Since the codes are randomly sampled from a Gaussian distribution, we have:%,
\begin{equation}
    \mathbb{E}\left[\bm{q}_k^T\bm{q}_k\right]=\bm{I}, \text{where}~\bm{q}\sim \mathcal{N}(0,1),
\end{equation}
which ensures that all elements of $\bm{W}$ are updated. As training progresses, the latent basis $\bm{W}$ converges to: \begin{equation}
    \lim_{t\rightarrow \infty}\bm{W}^{(t)} = \mathbb{E}_{z_e} \left[\bm{q}_k^T z_e\right]
\end{equation}
Thus, in the limit:
\begin{equation}
    \lim_{t\rightarrow \infty} \bm{q}_k \bm{W}^{(t)}=\mathbb{E}\left[\bm{q}_k\bm{q}_k^T\bm{e}\right]=\mathbb{E}\left[\bm{e}\right]
\end{equation}
At convergence, the product $\bm{q}_k \bm{W}$ equals the nearest feature. 



\begin{table*}[t]
\centering
\resizebox{1.85\columnwidth}{!}{
\begin{tabular}{lccccccc}
\toprule
 Method & Latent dim & Codebook size & Util$\uparrow$ & rFID$\downarrow$ & LPIPS$\downarrow$ & PSNR$\uparrow$ & SSIM$\uparrow$ \\
 \midrule
VQGAN \cite{Esser_2021_CVPR} & 128 & 65,536 & 1.4\% & 3.74 & 0.17 & 22.20 & 70.6 \\
VQGAN-EMA \cite{NEURIPS2019_5f8e2fa1} & 128 & 65,536 & 4.5\% & 3.23 & 0.15 & 22.89 & 72.3 \\
\midrule
VQGAN-FC \cite{yu2022vectorquantized}& 128 & 65,536 & 1.4\% & 5.33 & 0.18 & 21.45 & 68.8 \\
VQGAN-FC \cite{yu2022vectorquantized}& 8 & 65,536 & 100.0\% & 2.63 & 0.13 & 23.79 & 77.5 \\
FSQ$^\dagger$ \cite{mentzer2024finite} & 6 & 64,000 & 100.0\% & 2.80 & 0.13 & 23.63 & 75.8 \\
%LFQ \cite{yu2024language} & 65,536 & 100.0\% & 3.89 & 0.16 & 22.69 & 73.5 \\
LFQ \cite{yu2024language} & 16 & 65,536 & 100.0\% & 2.88 & 0.13 & 23.60 & 77.2 \\
VQGAN-LC-CLIP$^+$ \cite{Zhu2024ScalingTC} & 768 & 65,536 & 100.0\% & 2.40 & 0.13 & 23.98 & 77.3 \\
\midrule
SimVQ (ours) & 128 & 65,536 & 100.0\% & \textbf{2.24} & \textbf{0.12} & \textbf{24.15} & \textbf{78.4} \\
SimVQ (ours) & 128 & 262,144 & 100.0\% & \textbf{1.99} & \textbf{0.11} & \textbf{24.68} & \textbf{80.3} \\
\bottomrule
\end{tabular}
}
\caption{Reconstruction performance on ImageNet-1k with a resolution of $128\times 128$. All models are trained using images downsampled into $16 \times 16$ tokens. $\dagger$ Results are reproduced using the codebook size of $[8,8,8,5,5,5]$ to approximately match $65,536$. 
%$*$ LFQ is trained with a commitment loss coefficient of 0.25 for improved performance. 
$+$ Following VQGAN-LC, we extract CLIP features with the codebook frozen. The codebook utilization is calculated as the fraction of the codes that are activated at least once when encoding the validation set.} 
\label{tab:image_fid}
\end{table*}



\subsection{Efficiency Analysis}

SimVQ demonstrates greater efficiency than vanilla VQ due to its asymmetric training strategy, wherein the codebook $\bm{C}$ remains static and only the linear projection $\bm{W}$ is optimized. This approach results in a significant reduction in memory usage during the gradient backpropagation process. In vanilla VQ, the memory cost for codebook optimization is $O(Kd)$, where $K$ is the number of vectors in the codebook, and $d$ is the dimension of each vector. In our experiments, $K=65,536$ is much larger than $d=128$. As the vocabulary size increases, the memory required for backpropagation grows proportionally, significantly impacting resource consumption. In contrast, SimVQ’s memory cost for backpropagation is only $O(d^2)$ because the codebook $\bm{C}$ is fixed, and only the linear layer $\bm{W}$ is updated. This results in a constant memory requirement in backpropagation, independent of the vocabulary size. The $d \times d$ scaling becomes particularly advantageous as $K$ increases in practical applications. This structural design minimizes the computational overhead and improves training efficiency, especially when dealing with large vocabularies.



\section{Experiments}
To assess the efficacy and versatility of the proposed SimVQ, we conduct experiments across both image and audio modalities. Subsequently, we analyze the learned linear layer to investigate the latent basis. The experimental configurations are listed in Appendix \ref{appendix:config}.



\subsection{Vision Modality}



\subsubsection{Baselines}
Our baseline selection focuses on methods that enhance codebook utilization through architectural designs. We include VQGAN-FC \cite{yu2022vectorquantized}, FSQ \cite{mentzer2024finite}, LFQ \cite{yu2024language} and VQGAN-LC-CLIP \cite{Zhu2024ScalingTC} as our primary baselines, as they represent the current state-of-the-art in improving reconstruction performance through codebook architecture innovations. Traditional VQ variants such as RVQ \cite{Lee2022AutoregressiveIG} and MoVQ \cite{Zheng2022HighQualityPI} address fundamentally different technical challenges - they focus on training better discrete representations, rather than enhancing codebook utilization to improve reconstruction performance like our work. Another important research direction explores optimization-based solutions, where methods like stochastic quantization \cite{pmlr-v162-takida22a}, distribution penalty \cite{VQWasserstein,Xiao2023SCVAESC}, and codebook reset \cite{zheng2023online} tackle codebook utilization through sophisticated training strategies. Given our focus on architectural innovations in codebook design, we evaluate SimVQ against methods that share this technical foundation for a meaningful assessment of our contributions.

\subsubsection{Implementation Details}
To rigorously evaluate the proposed SimVQ, we reproduce all the VQ models listed in Tab. \ref{tab:image_fid} using the same architecture of VQGAN \cite{Esser_2021_CVPR} with the quantization layer different only.
Among the baselines, for VQGAN-FC \cite{yu2022vectorquantized}, we follow the original setting to reduce the dimension of the latent space to $8$ followed by $l_2$ normalization to improve codebook utilization. For FSQ \cite{mentzer2024finite}, we adopt a codebook size of $[8,8,8,5,5,5,]$ as recommended, to approximately match the default codebook size. For VQGAN-LC \cite{Zhu2024ScalingTC}, we follow them and leverage an external pre-trained CLIP model to extract features of the training dataset in advance for a well-defined latent space. All models are trained on the ImageNet \cite{5206848} dataset for 50 epochs with a batch size of 256. Input images are processed at a resolution of $128\times 128$ pixels and downsampled by a factor of $8$, yielding a feature map of $16\times 16 \times 128$, where $128$ is the dimension of the latent space. We set the default codebook size to a large number of $2^{16}=65536$ rather than the traditional number $8192$ to highlight the representation collapse problem. Performance is evaluated using rFID, LPIPS, PSNR, and SSIM metrics on the ImageNet validation set.



\begin{table*}[t]
\centering
\resizebox{1.85\columnwidth}{!}{
\begin{tabular}{lccccccc}
\toprule
 Method & Bandwidth & Util$\uparrow$ & UTMOS$\uparrow$ & PESQ$\uparrow$ & STOI$\uparrow$ & V/UV F1$\uparrow$ \\
 \midrule
GT & - & - & 4.06/3.48 & - & - & - \\
EnCodec \cite{efossez2023high} & 3.0kbps & - & 2.31/2.09 & 2.05/2.05 & 0.90/0.88 & 0.92/0.89 \\
Vocos \cite{siuzdak2024vocos} & 3.0kbps & - & 3.53/3.06 & 2.40/2.19 & 0.92/0.90 & 0.94/0.91 \\
SpeechTokenizer \cite{zhang2024speechtokenizer} & 3.0kbps & - & 3.56/3.02 & 1.93/1.74 & 0.88/0.84 & 0.93/0.89 \\
\midrule
WavTokenizer \cite{ji2024wavtokenizer} & 0.9kbps & 100/100\% & 3.74/3.43$^*$ & 2.01/2.26$^*$ & 0.89/0.89$^*$ & 0.92/0.92$^*$ \\
SimVQ (ours) & 0.9kbps & 100.0/100.0\% & \textbf{4.00/3.51} & \textbf{2.33}/2.08 & \textbf{0.91}/0.88 & \textbf{0.94}/0.91 \\
\midrule
WavTokenizer \cite{ji2024wavtokenizer} & 0.975kbps & 68/-\% & 4.02$^*$/- & 2.39$^*$/- & 0.92$^*$/- & 0.94$^*$/- \\
WavTokenizer \cite{ji2024wavtokenizer} & 1.05kbps & 27/-\% & 4.00$^*$/- & 2.36$^*$/- & 0.81$^*$/- & 0.94$^*$/- \\
SimVQ (ours) & 0.975kbps & 99.4/99.4\% & 4.03/3.52 & 2.42/2.15 & 0.92/0.88 & 0.94/0.92 \\
SimVQ (ours) & 1.2kbps & 99.4/99.0\% & 4.03/3.52 & 2.54/2.26 & 0.93/0.90 & 0.94/0.92 \\
SimVQ (ours) & 1.35kbps & 95.6/94.7\% & \textbf{4.03/3.53} & \textbf{2.61/2.31} & \textbf{0.93/0.90} & \textbf{0.95/0.93} \\
\bottomrule
\end{tabular}
}
\caption{Reconstruction performance on LibriTTS test-clean/test-other dataset. $*$ WavTokenizer is trained with a window size of 3 seconds. The bandwidth of 0.9kbps, 0.975kbps, 1.2kbps, 1.35kbps means the codebook size of 4096, 8192, 65536, 262144 respectively.}
\label{tab:audio_utmos}
\end{table*}




\subsubsection{Main Results}
Tab. \ref{tab:image_fid} presents the reconstruction performance of various VQ models on image data. We make three key observations: 1) Traditional VQGAN models utilize only a very small subset of the codebook, with a utilization rate of just $1.4\%$. Although VQGAN-EMA is proposed to improve codebook utilization, especially when the codebook size scales up to $65k$, it still suffers from severe representation collapse. 2) Recently proposed methods, such as LFQ, FSQ, and VQGAN-FC, effectively improve codebook utilization to $100\%$. However, these methods require reducing the latent space to a very low dimension. For example, applying VQGAN-FC to the standard latent dimension of $128$ results in severe representation collapse and degraded reconstruction performance. Additionally, these models face limitations in model capacity due to the low-dimensional latent space. While they achieve full codebook utilization, their reconstruction quality on rFID score lags significantly behind SimVQ. 3) VQGAN-LC-CLIP leverages an external pre-trained CLIP model to provide a well-defined latent space. However, VQGAN-LC relies on CLIP features pre-trained on much larger datasets than ImageNet, which introduces generalization issues and a lower performance ceiling (degradation issue in Tab. \ref{tab:ablation_codebook}). In contrast, SimVQ can be applied to a wide range of data types and achieves superior performance (rFID $2.40$ vs. $2.24$) without the limitations imposed by a pre-trained feature extraction model.


\begin{table}[t]
\centering
\resizebox{1.0\columnwidth}{!}{
\begin{tabular}{lcccccc}
\toprule
 Method & Codebook Size & Util$\uparrow$ & rFID$\downarrow$ & LPIPS$\downarrow$ & PSNR$\uparrow$ & SSIM$\uparrow$ \\
\midrule
VQGAN-LC-CLIP$^{\dagger}$ \cite{Zhu2024ScalingTC} & 50,000 & $99.9\%$ & 2.75 & 0.13 & 23.8 & 58.4 \\
VQGAN-LC-CLIP$^{\dagger}$ \cite{Zhu2024ScalingTC} & 100,000 & $99.9\%$ & \underline{2.62}& 0.12& 23.8 & 58.9 \\
VQGAN-LC-CLIP$^{\dagger}$ \cite{Zhu2024ScalingTC} & 200,000 & $99.8\%$ & \underline{2.66}& 0.12& 23.9 & 59.2 \\
 \midrule
SimVQ & 1,024 & 100.0\% & 3.67 & 0.16 & 22.34 & 70.8 \\
SimVQ & 8,192 & 100.0\% & 2.98 & 0.14 & 23.23 & 74.7 \\
SimVQ & 65,536 & 100.0\% & 2.24 & 0.12 & 24.15 & 78.4 \\
SimVQ & 262,144 & 100.0\% & \textbf{1.99} & \textbf{0.11} & \textbf{24.68} & \textbf{80.3} \\
\bottomrule
\end{tabular}
}
\caption{Ablation study on the effect of various codebook sizes on ImageNet at a resolution of $128 \times 128$. $\dagger$ We directly copy the reported results of VQGAN-LC from the original paper on ImageNet $256\times 256$ resolution.}
\label{tab:ablation_codebook}
\end{table}



\begin{table}[t]
\centering
\resizebox{1.0\columnwidth}{!}{
\begin{tabular}{ccccccc}
\toprule
 Initialization & Trainable & Util$\uparrow$ & rFID$\downarrow$ & LPIPS$\downarrow$ & PSNR$\uparrow$ & SSIM$\uparrow$ \\
 \midrule
Gaussian & Yes & 100.0\% & 2.31 & 0.12 & 24.04 & 77.2 \\
Uniform & No & 100.0\% & 2.31 & 0.12 & 24.15 & 78.4 \\
Gaussian & No & 100.0\% & \textbf{2.24} & \textbf{0.12} & \textbf{24.15} & \textbf{78.4} \\
\bottomrule
\end{tabular}
}
\caption{Ablation study of codebook optimization strategy.}
\label{tab:ablation_optim}
\end{table}









\subsubsection{Ablation Study}

\paragraph{On the Codebook Size} 
In Tab. \ref{tab:ablation_codebook}, we explore the impact of different codebook sizes, ranging from $1k$ to $262k$, which is typically the level of LLM's vocabulary size. SimVQ consistently improves performance as the codebook size increases. For instance, the rFID score decreases to $1.99$, and SSIM surpasses $80.0$. In contrast, while VQGAN-LC-CLIP can keep high codebook utilization as increasing codebook size, it encounters performance degradation, with the rFID score worsening from $2.62$ to $2.66$ when the codebook size is increased from $100,000$ to $200,000$. 


\paragraph{On the Codebook Optimization Strategy}
We investigate codebook initialization and the training of the linear layer in Tab. \ref{tab:ablation_optim}. Our findings are as follows: 1) The codebook is robust to different initialization strategies, yielding similar results with both Gaussian and uniform initialization. 2) When the codebook is updated during training, SimVQ continues to address the representation collapse issue, though with a slight degradation in performance.





\subsection{Audio Modality}


\subsubsection{Baselines and Implementation Details}

We use the LibriTTS dataset \cite{zen2019libritts} for audio-based VQ model training.  The baselines such as Encodec \cite{efossez2023high}, Vocos \cite{siuzdak2024vocos}, and SpeechTokenizer \cite{zhang2024speechtokenizer} are based on residual vector quantization method. Our SimVQ model adopts the same architecture as WavTokenizer \cite{ji2024wavtokenizer} with the only modification being the replacement of their EMA codebook with our one linear layer reparameterization method. We train SimVQ on LibriTTS-580h for 50 epochs with a batch size of 64. Note that WavTokenizer is trained with a 3-second window size for optimal performance, we train SimVQ using a 1-second window to accelerate training. For objective evaluation of the reconstructed audio, we follow Vocos \cite{siuzdak2024vocos} and employ metrics such as UTMOS \cite{Saeki2022UTMOSUS}, PESQ \cite{941023}, STOI, and the F1 score for voiced/unvoiced classification (V/UV F1). UTMOS is particularly valuable as it produces scores highly correlated with human evaluations.

\subsubsection{Main Results}
Tab. \ref{tab:audio_utmos} presents the reconstruction performance of various VQ models on audio data. Baseline models using residual vector quantization perform significantly worse than SimVQ, even when utilizing much larger bandwidths. Despite using the same architecture as WavTokenizer, our model, which replaces the quantization layer with SimVQ, achieves superior performance with a 1-second window size and maintains nearly $100\%$ codebook utilization when scaling up to a size of 262,144. The consistent performance of the SimVQ model across both image and audio data demonstrates that SimVQ is a general method for addressing the representation collapse problem in VQ models and can be effectively applied across multiple modalities.



\subsection{Analysis}

In Fig. \ref{fig:rank_norm}(a), we plot the rank of the latent basis matrix over training epochs. Notably, SimVQ demonstrates the ability to adaptively adjust the rank of the latent space. Specifically, when the codebook size increases from $65k$ to $262k$, the rank of the latent basis matrix decreases more rapidly and converges to a lower value. This observation suggests that a larger codebook can effectively alleviate the pressure on the latent space dimensionality, allowing the model to learn to represent data more efficiently. Additionally, despite the rank decreasing to a lower-rank space, SimVQ maintains $100\%$ codebook utilization, highlighting its superiority over VQGAN-FC, which struggles when increasing the latent dimension from 8 to 128.
We also calculate the Frobenius norm of the latent basis matrix, as shown in Fig. \ref{fig:rank_norm}. The norm of a codebook size of $262k$ is slightly larger than for $65k$, indicating that a larger codebook can span a broader area in the linear space.
For a comprehensive evaluation, we also provide the reconstruction loss curve on the ImageNet validation dataset in Appendix \ref{appendix:loss}. The results consistently show that SimVQ achieves improved performance, further validating the effectiveness of our approach.



\begin{figure}[t]
    \centering
    \begin{minipage}[b]{0.49\columnwidth}
        \centering
        \includegraphics[width=\columnwidth]{material/w_rank_65k_262k.pdf}
    \end{minipage}
    %\hspace{0.05\textwidth}
    \begin{minipage}[b]{0.49\columnwidth}
        \centering
        \includegraphics[width=\columnwidth]{material/w_norm_65k_262k.pdf}
    \end{minipage}
    \caption{(a):(left) The rank of latent basis matrix $\bm{W}$ over training epochs. (b):(right) The Frobenius norm of latent basis matrix $\bm{W}$ over training epochs.}
    \label{fig:rank_norm}
\end{figure}


\begin{figure}[t]
    \centering
    \includegraphics[width=0.8\columnwidth]{material/combined_vq_visualization_part_codebook.pdf}
    \caption{Visualization of the divergence between the encoder features and codebook embeddings on a random subset of ImageNet validation dataset. The left figure is vanilla VQ model and the right one is SimVQ.}
    % \caption{Visualization of the divergence between the encoder features and codebook embeddings on a random subset of ImageNet validation dataset. (left) Results for vanilla VQ; (right) Results for SimVQ.}
    \label{fig:visualization}
\end{figure}




\subsection{Qualitative Evaluation}
We visualize the distribution of encoder features and codebook embeddings in Fig. \ref{fig:visualization} and the frequency in Appendix \ref{appendix:freq}. Compared to vanilla VQ models that most codebook vectors are not unused, SimVQ can update the whole codebook and distribute the features evenly in the whole space.
We qualitatively compare the reconstruction quality of both images and audio in Appendix \ref{appendix:cases}. SimVQ can presreve more details with an enlarged codebook size, such as ``eyes'' and ``text'', which are challenging for vanilla VQ models. 






\section{Discussion}


\paragraph{VQ Performance and Generative Models}


Many generative models \cite{Rombach_2022_CVPR,yu2022scaling,sun2024autoregressive,team2024chameleon} utilize VQ models as a ``tokenizer'' to obtain discrete tokens. While it is intuitive that VQ reconstruction quality should impact generation performance, recent studies \cite{zhu2024stabilize,yu2024language} have revealed that the relationship is more nuanced: improved VQ reconstruction metrics do not necessarily translate to better generative outcomes. This complex relationship suggests that evaluating VQ models primarily through downstream generation tasks may not provide the most insightful assessment of their fundamental properties. Therefore, we focus on addressing the critical issue of representation collapse in VQ models, aiming to advance our understanding of their core representation learning mechanisms.

\section{Conclusion}
In this paper, we explore the representation collapse problem in VQ models. We conduct a theoretical analysis of the optimization process in VQ models and propose a simple yet effective method, SimVQ, to address this issue. Our method addresses the representation collapse by jointly optimizing the latent space through a linear transformation with one linear layer. Experimental results demonstrate that SimVQ outperforms previous approaches on both image and audio datasets, highlighting its broad applicability across diverse machine learning tasks.


\section*{Acknowledgement}

This research was supported by the National Natural Science Foundation of China (Grant No.62276245).